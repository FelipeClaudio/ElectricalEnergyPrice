 \paragraph{}Neste trabalho foi proposto um modelo baseado em redes neurais, com o objetivo de prever o valor m�dio do PLD para alguns meses a frente. Para isso, foi necess�ria a decomposi��o do sinal, deixando para o treinamento da rede somente a parte n�o-determin�stica. Foram testados diferentes m�todos de extra��o das componentes at� chegar na que foi utilizada no Cap�tulo 5. Al�m disso, as MLPs foram treinadas para quantidades variadas de neur�nios na camada intermedi�ria, buscando a melhor arquitetura que solucionasse o problema proposto.
 
 \paragraph{}Diante dos resultados obtidos com o trabalho, observou-se a dificuldade em utilizar redes neurais MLP para conjuntos de dados pequenos. Apesar disso, atingiu-se os objetivos mencionados no come�o do projeto. 
  
 \paragraph{}O m�todo utilizado para o pr�-processamento dos dados pode ser replicado para outros tipos de s�ries temporais, assim como o m�todo para a sele��o da melhor rede pode ser explorado em outros trabalhos.
 
 \paragraph{}Quanto �s conclus�es obtidas atrav�s do experimento, observou-se que o erro tem uma tend�ncia crescente conforme o n�mero de meses a frente na previs�o, assim como esperado, tornando cada vez mais complexa a utiliza��o do modelo para cen�rio muito distantes. A aplica��o da rede para corre��o dos erros trouxe para a previs�o um compromisso entre erro m�ximo e m�dio, conforme a arquitetura escolhida.
 
 \paragraph{}Para trabalhos futuros, seria interessante comparar os resultados obtidos com modeos diferentes, como por exemplo, SARIMA e GARCH e ADALASSO. Al�m disso, podem ser exploradas outras formas de extra��o das componentes da s�rie temporal, de forma a obter um sinal residual com menos energia dos que os que foram encontrados.
 
 \paragraph{}Outra abordagem pode ser a de considerar os dados para os anos anteriores 2015 e analisar quanto isso afeta a previs�o do PLD m�dio mensal. Nesse caso, cabe tamb�m utilizar estruturas de redes neurais mais complexas, como por exemplo LSTM - \textit{Long Short-Term Memory}.
